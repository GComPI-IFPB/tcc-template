\chapter[Introdução]{Introdução}
\label{cap:intro}
% \addcontentsline{toc}{chapter}{Introdução}

PRINCIPAIS PROBLEMAS:

\begin{itemize}
  \item definição do tema e do problema
  \item não se delimita o que é estudado, o corte da pesquisa e o problema
  \item mostrar só “APLICAÇÃO DA CIÊNCIA DA COMPUTAÇÃO EM ALGO”, mas o trabalho com o tema deve contribuir para uma divisão temática da área. Por exemplo:
\end{itemize}

SUGESTÃO DE ROTEIRO:
\begin{enumerate}
  \item Contextualização do assunto com sua importância, significado para a área ligada com seu estudo, atualidade etc.\\ \\ “Nos dias de hoje, um dos assuntos mais discutidos na área de redes tem sido...” \\ “Atualmente, o emprego de metodologias ativas com o uso de tecnologia tem...”\\ “Pesquisas sobre visualização de dados são muito relevantes para o desenvolvimento de aplicações em XXXX, pois...”\\
  \item CUIDADO!!!! Afirmações categóricas devem ter referências. Por exemplo, se você começar dizendo que a área de machine learning é muito importante para o diagnóstico médico, precisa citar um autor que dê respaldo.
  \item Conceitos (poucos – uma a três citações porque serão mais bem desenvolvidos na fundamentação teórica)
  \item Mostre como os autores abordam problemas relacionados ao assunto que você vai tratar
  \item Identifique lacunas, incompletudes, situações específicas e melhorias que poderiam ser feitas no modo como os autores têm tratado o problema
  \item Mostre como vai ATACAR UM PROBLEMA! Você pode dizer que um autor deixou de lado algo importante, que pode complementar uma solução, que pode abordar um problema de modo mais específico para uma situação particular etc. Aqui temos o “LEITMOTIV” do seu trabalho: um norte que deve guiar você em cada autor que for usar, metodologia que for aplicar, modos de analisar etc.
\end{enumerate}
% --------------------------------------


\section{Justificativa e Relevância do Trabalho}
\label{sec:justificativa}
Explicar porque foi escolhido esse tema, qual sua importância para a área, etc...


% --------------------------------------


\section{Objetivos}
\label{sec:objetivo}

\subsection{Objetivo Geral}
\label{sec:objgeral}
O objetivo principal deste trabalho ......


\subsection{Objetivos Específicos}
\label{sec:objespecificos}

\begin{itemize}
  \item XXXXXXXXXX

  \item XXXXXXXXXXX

\end{itemize}

% --------------------------------------


\section{Metodologia}
\label{sec:metodologia}

A metodologia empregada para desenvolvimento do trabalho ....


%% ------------------------------------------------------------------------- %%


\section{Organização do Documento}
\label{sec:organizacao}

No Capítulo~\ref{cap:capitulos}, são apresentados os conceitos relacionados .........

No Apêndice \ref{ape:apendiceI}, é apresentada a base de dados utilizada no trabalho.....
